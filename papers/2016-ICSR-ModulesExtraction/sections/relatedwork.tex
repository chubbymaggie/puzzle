\section{Related work and discussion}
\label{sec:relatedwork}

Leveraging reuse in the construction of DSLs is an objective that has been previously discussed in the research community on software languages engineering. One of the very first achievements towards this objective was the notion of components-based language development. With the time passing, approaches are becoming more sophisticate, thus supporting more complex modularization scenarios, and being applicable for more diverse technological spaces \cite{Mernik:2013,Rumpe:2010,Voelter:2013b}. 

More recent approaches are focused not only on dealing with modularization issues, but also on facilitating the reuse process itself. For instance, Melange \cite{Degueule:2015} is a tool-supported approach that introduces some operators (such as slice, inheritance, and merge) intended to manipulate legacy DSLs in such a way that they can be easily integrated in new developments. Using Melange, a language designer can combine a set of DSLs in different ways to produce a new one.

The main contribution of our approach is that it advances towards the automation of the reuse process. We demonstrate that, with the correct abstractions and assuming some constraints, the process can be automated by means of reverse-engineering techniques. It is important mentioning that our approach uses some of the ideas presented by Caldiera and Basili \cite{Caldiera:1991}. That approach proposes reverse-engineering methodologies for extracting reusable modules in objects oriented software. In addition, our work is based on an observation about commonalities and potential reuse provided by V\"oelter et al \cite[p. 60-61]{voelter:2013}. We show that the notion of commonalities is quite useful for extracting reusable language modules.

There are, however, some open issues that need further investigation. In particular, during the conduction of this research, and trying to apply the approach in further case studies, we realized that the comparison operators to detect those commonalities can become an Achilles' heel. In some cases, the notion of commonality can be associated to a given \textit{functionality} (abstractly speaking), more than equality in the specification. For example, there are many DSLs that use constraints languages but that use different language constructs to support them. They share the functionality of constraint languages, but the specifications do not match. That does not mean that there is not potential reuse. The detection of this kind of commonalities can become quite difficult due to the ambiguity in that notion of functionality.

As a matter of fact, considering more flexible approaches for the detection of commonalities can have additional advantages. Consider for example two DSL that define different constraints languages where one is better defined (more completely or with more powerful capabilities) than the other. If this situations can be detected, and language designers can chose a preferred language module, our approach can become useful not only to achieve reuse but also to improve quality of existing DSLs. We claim that more complex overlapping identification (probably with human intervention) should be provided.

%Software reuse has been largely studied during the last decades. The complexity behind the definition of reusable software is well-known and many approaches has been proposed to facilitate this task. The work of Caldiera and Basili \cite{Caldiera:1991} is a clear example of such approaches. It is intended to automatically extract reusable software modules from existing software systems. Then, a set of metrics for qualifying the quality of the produced modules. As mentioned in the introduction, this work has inspired the approach presented in that work paper. 

%The main contribution of the research presented this paper is that we apply those ideas in the construction of DSLs. To do so, we use a strategy based on Venn diagrams that, in turn, has been inspired in the phenomenon of domains overlapping identified by .

%As a matter of fact, our approach is not the first one that tries to increase reuse in the construction of DSLs. The community of software languages engineering has been intensively working on this issue and noways we can find approaches for components-based languages development (such as ). In that context, our approach can be positioned as a reverse engineering technique for increasing reuse where reusable language modules are extracted from existing DSLs. It is worth noting that there are other approaches working on reverse engineering for DSLs. For example, the research presented in \cite{vacchi:2014} and \cite{Kuhn:2015}, is intended to synthesize language product lines (i.e., software product lines where the products are DSLs) from a DSL specification. 

 

