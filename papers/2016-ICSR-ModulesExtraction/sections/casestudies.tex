\section{Evaluation}
\label{sec:validation}

In this section we evaluate our approach by using two different evaluation scenarios. The former corresponds to a case study of a set of DSLs for expressing state machines. The idea is to test our approach in a set of language where we know that there are commonalities that we know in advance. The second case study aims to test our approach in a more realistic scenario. We take some DSLs from GitHub public repositories. 



\begin{table*}[htbp]
  \centering
 \scalebox{0.8}{
\begin{tabular}{|p{0.2\textwidth}|p{0.3\textwidth}p{0.1\textwidth}p{0.4\textwidth}|}
\hline
\multicolumn{4}{|c|}{\textbf{Hypotheses of Experiment 1}} \\ \hline
\textbf{Null Hypothesis ($H_0$)} & \multicolumn{ 3}{|p{0.8\textwidth}|}{\toolname is capable of detecting commonalities in the case study that motivated this research.} \\ \hline
\textbf{Alt. Hypothesis ($H_1$)} & \multicolumn{ 3}{|p{0.8\textwidth}|}{
\toolname is not capable of detecting commonalities in the case study that motivated this research.} \\ \hline
\textbf{Dependent variable} & \multicolumn{ 3}{|p{0.8\textwidth}|}{The set of ecores representing our languages. }\\ \hline
\textbf{Blocking variables} & \multicolumn{ 3}{|p{0.8\textwidth}|}{The most sold phones and the market share indexes. }\\ \hline
\textbf{Model used as input} & \multicolumn{ 3}{|p{0.8\textwidth}|}{\textit{models in \url{urlhacialosmodelos}} }
\\
\hline \hline


\multicolumn{4}{|c|}{\textbf{Hypotheses of Experiment 2}} \\ \hline
\textbf{Null Hypothesis ($H_0$)} & \multicolumn{ 3}{|p{0.8\textwidth}|}{The use of \toolname 
will not result in a higher market-share impact metric than selecting the most commonly sold 
phones, for a given maximum budget.} \\ \hline
\textbf{Alt. Hypothesis ($H_1$)} & \multicolumn{ 3}{|p{0.8\textwidth}|}{The use of \toolname 
will result in a higher market-share impact metric than selecting the most commonly sold 
phones, for a given maximum budget.}\\ \hline
\textbf{Model used as input} & \multicolumn{ 3}{|p{0.8\textwidth}|}{\textit{Android feature model presented in Figure \ref{fig:featureModel}} }\\ 
\hline
% @J - Do you mean independent?! My understanding is that a blocking variable is a grouping variable...
\textbf{Blocking variables} & \multicolumn{ 3}{|p{0.8\textwidth}|}{The most sold phones, market share indexes and the maximum cost allowed set to 600\$. }\\ \hline
\textbf{Model used as input} & \multicolumn{ 3}{|p{0.8\textwidth}|}{\textit{Android feature model presented in Figure \ref{fig:featureModel}} }\\
\hline \hline

\multicolumn{4}{|c|}{\textbf{Constants}} \\ \hline
\textbf{CSP solver} & \multicolumn{ 3}{|p{0.8\textwidth}|}{\textit{ChocoSolver v2} } \\ \hline
\textbf{Heuristic for variable selection in the CSP solver} & \multicolumn{ 3}{|p{0.8\textwidth}|}{\textit{Default}}\\
\hline 

\hline 
\end{tabular}%
}
\caption{Hypotheses and design of experiments.}
  \label{tab:Exp1aDesign}
\end{table*}
\todo{poner la tabla para con los datos de los experimentos que vamos a ejecutar/hemos ejecutado. Intenta pensar cuales pueden ser las conclusiones que quieres extraer. Yo propongo 3 abajo.}

Table \ref{tab:Exp1aDesign} shows the hypothesis of the experiments executed to validate our 
approach. To make the experiments reproducible, a number of fixed assumptions are made, such as homogeneous feature costs. ChocoSolver 
\footnote{\url{http://www.emn.fr/z-info/choco-solver/}}, with it's default heuristic, is 
used as the CSP solver for extracting software products from the feature model presented 
in Figure \ref{fig:featureModel}

\textbf{Technological space and experimental platform:} Currently, there are diverse techniques available for the implementation of syntax and semantics of DSLs \cite{Mernik:2005b}. Language designers can, for example, choose between using context-free grammars or metamodels as specification formalism for syntax. Similarly, there are at least three methods for expressing semantics: operationally, denotationally, and axiomatically \cite{Mosses:2001}. In this paper we are interested on DSLs which syntax is specified by means of metamodels and semantics is specified operationally as a set methods (a.k.a, \textit{domain-specific actions} \cite{Combemale:2013}). Each language construct is specified by means a metaclass and the relationship between language constructs are specified as references between metaclasses. In turn, domain-specific actions are specified as java-like methods that are allocated in each metaclass. The experiments were 
conducted using a version of \toolname implemented in Java. Further, 
\toolname was installed in the Grid5000 Cloud, which is a cluster with more than 5000 cores from were we took XX dual-CPU Dell Blades with
Intel Xeon X3470 CPUs running at 2.93GHz, with 16 threads 
per CPU, and CentOS v6. Each dual-CPU Dell Blade has 36GB of RAM. 



\subsection{Experiment 1: ATOS industrial project}

\subsection{Experiment 2: Identifying potential reuse in the wild}
 
\subsection{Experiment 3: Scalability of our solution}
