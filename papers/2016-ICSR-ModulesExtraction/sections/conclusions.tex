\section{Conclusions}
\label{sec:conclusions}
%lessons learned????
%\todo[inline]{eventualmente tenemos que extender esta parte... aunque no hay mucho sitio... escribela si peudes y ya vemos que hacemos... si no para la extension de revista ;)}
In this paper, we presented an approach to exploit reuse during the construction of DSLs. We show that it is possible to partially automate the reuse process by identifying overlapping among DSLs and automatically extracting reusable language modules that can be later used in the construction of new DSLs.

We evaluated our approach in a real industrial case study and we demonstrate that there is an important amount of potential reuse in DSLs in public repositories. More concretely, based on empirical data, we showed that in about the \textbf{50\%} of the new DSLs there are reuse opportunities to exploit reuse. This reuse is, in average, about the \textbf{18\%} of the size of the DSLs. 

%As future work, we plan to propose approaches to automatically build language product lines i.e., software product lines where the products are DSLs. The intention is to follow with the idea of automating the reuse process. This time, using ideas that facilitate the management of the variability existing among DSLs. 