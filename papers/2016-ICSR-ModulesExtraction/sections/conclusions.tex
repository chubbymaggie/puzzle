\section{Conclusions and Perspectives}
\label{sec:conclusions}

The use of DSLs has demonstrated advantages in the software development process. However, building a DSL is a costly task that requires important engineering efforts. In this paper, we provide an approach for exploiting reuse during the construction of DSLs. We demonstrate that it is possible to partially automate the reuse process by identifying commonalities among DSLs and automatically extracting reusable language modules that can be later used in the construction of new DSLs. We evaluated our approach in a real industrial case study and we demonstrate that there is an important amount of potential reuse in DSLs in public repositories.

As future work, we plan to propose approaches to automatically build language product lines i.e., software product lines where the products are DSLs. The intention is to follow with the idea of automating the reuse process. This time, using ideas that facilitate the management of the variability existing among DSLs. 