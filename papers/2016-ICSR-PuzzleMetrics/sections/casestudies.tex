\section{Validation}
\label{sec:validation}

In this section we present the validation our approach which is performed into two phases. In a first phase, we take a set of DSLs existing in the literature, we implemented it in kermeta, and we perform the corresponding analysis. The idea is to show how our approach behaves in presence of a set of DSLs that we know shares commonalities and where we alredy know the existing semantic variability. A second phase is a more general one. In this case we want to detect potential reuse in existing DSLs that we can find in GitHub repositories. Because kermeta is a quite new benchmark, we there are not many DSLs. However, we can find many metamodels. 

\subsection{Rhapsody + Harel's Statecharts + UML}

An important of this family with respect to the other two is that whereas in the two .. cases the families were built by us by respecting some knowledge of the... this other goes from a different team. The code of this family already exists. This allow us to see the behavior of our tool in a real-life example thus reducing the possible slant that may exist during the construction of the example. 

\subsection{Identifying potential reuse in DSLs from GitHub}


