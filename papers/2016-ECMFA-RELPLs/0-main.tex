\documentclass[runningheads,a4paper]{llncs}

\usepackage{amssymb}
\usepackage{amsmath}
\usepackage{graphicx}
\usepackage[applemac]{inputenx} % This is required to support characters in spanish
\usepackage[T1]{fontenc} % This is required to have small-bold-capitals
\usepackage{fancyhdr} % This is required to have nice headers and footers
\usepackage{framed} % This is required to have framed comments
\usepackage{color} % This defines colors that can be used in the text
\usepackage{soul} % This allows the usage of \hl to highlight text
\usepackage{tocloft} % This is required to have custom lists of .....
\usepackage{comment} % This is required to remove specific environments
\usepackage{caption}
\usepackage{ifthen}
\usepackage{array}
\usepackage{wrapfig}
\usepackage{hyperref}
\usepackage{pifont}
\usepackage{listings}
\usepackage{syntax}
\usepackage{graphicx}
\usepackage{url}
\usepackage{latexsym}
\usepackage{fancybox}
\usepackage{pifont}
\usepackage{lineno}
\usepackage{todonotes}
\usepackage{multirow}
\usepackage{multicol}
\usepackage{xspace}
\usepackage{setspace}
\usepackage{todonotes}
\usepackage{bbding}
\usepackage{xcolor,colortbl}
\setcounter{tocdepth}{3}
\newtheorem{mydef}{Definition}


\newcommand{\mc}[2]{\multicolumn{#1}{c}{#2}}
\definecolor{Gray}{gray}{0.9}
\definecolor{LightGray}{gray}{0.98}

\newcolumntype{a}{>{\columncolor{Gray}}c}
\newcolumntype{b}{>{\columncolor{white}}c}

\newcommand\etal[0]{\emph{et al.}\xspace}
\newcommand\td[1]{\todo[inline]{#1}\xspace}

\setcounter{tocdepth}{3}


\urldef{\mailsa}\path|{david.mendez-acuna,jagalindo,benoit.combemale,benoit.baudry}@inria.fr|    
\newcommand{\keywords}[1]{\par\addvspace\baselineskip
\noindent\keywordname\enspace\ignorespaces#1}

\begin{document}

\mainmatter  % start of an individual contribution

\title{Reverse Engineering Language Product Lines}
\titlerunning{Reverse Engineering Language Product Lines}


\author{David M\'endez-Acu\~na \and Jos\'e A. Galindo \and Benoit Combemale \and Benoit Baudry}
\institute{University of Rennes 1, INRIA/IRISA. France\\
\vspace{1mm}\mailsa}
\authorrunning{David M\'endez-Acu\~na et. al}

\maketitle

\begin{abstract} 

The use of domain-specific languages (DSLs) is becoming a successful technique in the implementation of complex systems. However, the construction of this type of languages is time-consuming and requires highly-specialized knowledge and skills. Hence, researchers are currently seeking approaches to leverage reuse during the DSLs development in order to minimize implementation from scratch. An important step towards achieving this objective is to identify commonalities among existing DSLs. These commonalities constitute potential reuse that can be exploited by using reverse-engineering methods. In this paper, we present an approach intended to identify sets of DSLs with potential reuse. We also provide a mechanism that allows language designers to measure such potential reuse in order to objectively evaluate whether it is enough to justify the applicability of a given reverse-engineering process. We validate our approach by evaluating a large amount of DSLs we take from public \texttt{GitHub} repositories.

\end{abstract}

\section{Introduction}
\label{sec:introduction}

% Research context
The use of domain-specific languages (DSLs) has become a successful technique to achieve separation of concerns in the development of complex systems \cite{Cook:2006}. A DSL is a software language in which expressiveness is scoped into a well-defined domain that offers a set of abstractions (a.k.a., language constructs) needed to describe certain aspect of the system \cite{Combemale:2014}. For example, in the literature we can find DSLs for prototyping graphical user interfaces \cite{Oney:2012}, specifying security policies \cite{Lodderstedt:2002}, or performing data analysis \cite{Eberius:2012}.
 
\vspace{3mm}
Naturally, the adoption of such language-oriented vision relies on the availability of the DSLs needed for expressing all the aspects of the system under construction \cite{Clark:2013}. This fact carries the development of many DSLs which is a challenging task due the specialized knowledge it demands. A language designer must own not only quite solid modeling skills but also the technical expertise for conducting the definition of specific artifacts such as grammars, metamodels, compilers, and interpreters. As a matter of fact, the ultimate value of DSLs has been severely limited by the cost of the associated tooling (i.e., editors, parsers, etc...) \cite{jezequel:2014}.

To improve cost-benefit when using DSLs, the research community in software languages engineering has proposed mechanisms to increase reuse during the construction of DSLs. The idea is to leverage previous engineering efforts and minimize implementation from scratch \cite{Storm:2013}. These reuse mechanisms are based on the premise that ``software languages are software too'' \cite{Favre:2011} so it is possible to use software engineering techniques to facilitate their construction \cite{Kleppe:2009}. For instance, there are approaches that take ideas from Component-Based Software Engineering (CBSE) \cite{Cleenewerck:2003} and Software Product Lines Engineering (SPLE) \cite{Zschaler:2010} during the construction of new DSLs.

The basic principle underlying the aforementioned reuse mechanisms is that language constructs are grouped into interdependent \textit{language modules} that can be later integrated as part of the specification of future DSLs. Current approaches for modular development of DSLs (e.g., \cite{Vacchi:2015,Mernik:2013,Rumpe:2010}) are focused on providing foundations and tooling that allow language designers to explicitly specify dependencies among language modules as well as to provide the composition operators needed during the subsequent assembly process.

% Problem statement
Despite such important advances, the definition of language modules that can be actually useful in future DSLs is still a difficult task. This difficulty is due to several factors. For example, the reuse of a language module implies the reuse of all the constructs it offers; in many cases, some of those constructs are not necessary and, worst, they might be even conflictive. Language modules rarely can be reused `as-is' and without any adaptation. In this context, our research question is: \textit{How to define language modules that can be actually reusable in the construction of future DSLs?} %To answer that question language designers have to deal with several issues. For example, they have to find an appropriated level of granularity where, on one hand, language modules are fine enough so they contain small sets of constructs that are useful in several situations; on the other hand, language modules that are too small may over-complex the definition of DSLs so a good balance should be found. In addition, language designers need to find the correct combination of constructs that go well together.

Inspired in the work of Caldiera and Basili \cite{Caldiera:1991}, in this paper we propose a tool-supported approach that analyses a set of existing DSLs to extract a catalog of reusable language modules. The main idea is to perform static analysis on a given set of DSLs that are implemented in an homogeneous technology in order to detect groups of constructs that are typically used together. Once those groups are identified, we extract them by means of a break-down algorithm. Our approach considers not only syntax but also semantics of the DSLs under study. We validate our approach by applying it in an evaluation scenario with realistic DSLs that we obtain from public \textsc{GitHub} repositories. The results of this validation are quite promising.

%In this paper we propose an approach to leverage reuse during the construction of DSLs focused on legacy DSLs. In other words, we propose to exploit reuse in existing DSLs that are were not necessarily built for being reused but that share some commonalities (i.e., they provide similar language constructs). In particular, we present an strategy to identify groups of constructs that are frequently used together in a given sets of legacy DSLs. Then, a catalog of language modules is extracted from the commonalities by means of a reverse-engineering method. To do so, we perform static analysis on the artifacts where the DSLs are specified and compare language constructs at the level of the syntax and semantics.

The reminder of this paper is organized as follows: Section \ref{sec:background} introduces a set of preliminary definitions/assumptions as well as an illustrating scenario that we use all along the paper. Section \ref{sec:apprach} describes our approach that is evaluated in Section \ref{sec:validation}. Section \ref{sec:relatedwork} presents the related work and, finally, Section \ref{sec:conclusions} concludes the paper. 

%The aforementioned reuse mechanisms can be adopted by means of two different approaches: \textit{top-down} and \textit{bottom-up}. The top-down approach proposes the design and implementation from scratch of reusable language modules that can be employed in the construction of several DSLs. Differently, the bottom-up approach proposes the use or reverse-engineering processes to extract those language modules from existing DSLs that share some commonalities which can be properly encapsulated. Whereas the major complexity in top-down approaches is that language designers should design language modules by trying to predict their potential reuse; bottom-up approaches do not have to deal with that problem. Rather, the extraction of the reusable language modules is based on the detection of commonalities in existing DSLs. Consequently, bottom-up approaches can be considered as promising approach and, indeed, there are already in the literature some works (e.g., \cite{vacchi:2014}) advancing towards that direction. 

%The success of this strategy relies on a set of design decisions that favor extensibility and/or genericity thus increasing the probabilities that the language modules are useful in the future. In this case, the major complexity comes from the fact that language designers do not know \textit{a priori} the needs of future DSLs. Consequently, in practice many of these building blocks are not reusable \textit{as is} and rather they require some previous adaptation.

% Contribution


%It is quite important to mention that our approach is inspired on the work of Berger et al X that presents the foundations for measuring product line-ability of families of software products in the general case. Our contributions with respect to that work are that we apply these ideas to the particular case of DSLs. Besides, at the best of our knowledge, there is not a tool implementing such ideas so we introduce a proposal. Finally, we apply our proposal in an empirical study performed on GitHub repositories where the objective is to identify families of DSLs in the Web and to evaluate its potential reuse. 

% Outline


%Despite such important advances, the definition of language modules that can be actually useful in future DSLs is still a difficult task. In part, this difficulty is due to the fact that the reuse of a language module implies the reuse of all its constructs and language designers do not always have the information that allow them to predict the correct combination of constructs that go well together. What is the correct level of granularity? Are there constructs that should be always together? Are there constructs that should be always separated?

\section*{Acknowledgments}
The research presented in this paper is supported by the European Union within the FP7 Marie Curie Initial Training Network ``RELATE" under grant agreement number 264840 and VaryMDE, a collaboration between Thales and INRIA.

% BIBLIOGRAPHY
\bibliographystyle{abbrv}
\bibliography{0-main}

\end{document}
